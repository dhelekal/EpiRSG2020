\documentclass{article}

% set font encoding for PDFLaTeX, XeLaTeX, or LuaTeX
\usepackage{ifxetex,ifluatex}

\if\ifxetex T\else\ifluatex T\else F\fi\fi T%
  \usepackage{fontspec}
\else
  \usepackage[T1]{fontenc}
  \usepackage[utf8]{inputenc}
  \usepackage{lmodern}
\fi

\usepackage{hyperref}
\usepackage[parfill]{parskip}
\usepackage{amsmath}
\usepackage{amssymb}
\usepackage{amsthm}
\usepackage{mathtools}
\usepackage{physics}
\usepackage{enumitem}
\usepackage{multicol}
\usepackage{graphicx}
\usepackage{lipsum}
\usepackage[export]{adjustbox}
\title{Rubella Report}
\author{David Helekal, Yiping Zhang}

\begin{document}
\maketitle
\section{Model}

Let us define a $k$-age group compartmental SIRV model for the infection dynamics of the diseases covered by the MMR vaccine. This model takes the form of a system of four vectorised ordinary differential equations. Each of the $k$ vector entries then corresponds to a given age subgroup with span $m_i, \quad i\in1...k$. Each equation then governs the dynamics what proportion of the subpopulations is susceptible, infected, recovered, or has been vaccinated.

\begin{align*}
&\frac{d\mathbf{s}}{dt} &=& \mathbf{B} - \mathbf{V}\mathbf{s} - \mathbf{D}\mathbf{s} - \beta(t)\mathbf{s}*\mathbf
{i}+\delta_{t_{end}}(t)\mathbf{M}\mathbf{s}\\
&\frac{d\mathbf{i}}{dt} &=&\beta(t)\mathbf{s}*\mathbf{i} - \mathbf{D}\mathbf{i} - \mathbf{\gamma}\mathbf{i}+\delta_{t_{end}}(t)\mathbf{M}\mathbf{i}\\
&\frac{d\mathbf{r}}{dt} &=& \gamma\mathbf{i} - \mathbf{D}\mathbf{r}+\delta_{t_{end}}(t)\mathbf{M}\mathbf{r}\\
&\frac{d\mathbf{v}}{dt} &=& \mathbf{V}\mathbf{s}-\mathbf{D}\mathbf{v} +\delta_{t_{end}}(t)\mathbf{M}\mathbf{v}\\
\end{align*}
$\mathbf{s}*\mathbf{i}\quad\textit{Denotes the pointwise product}$
Where:
\begin{align*}
&\mathbf{s}, \mathbf{i}, \mathbf{r} ,\mathbf{v}\in \mathbb{R}^k& \quad& \text{are k-age group classes of susceptible, infected, recovered}\\
&\mathbf{B} \in \mathbb{R}^{k\times k}, \mathbf{B}:=diag(B,0,...,0)& \quad& \text{Birth rate}\\
&\mathbf{V} \in \mathbb{R}^{k\times k}, \mathbf{V}:=diag(V_1,...,V_k) &\quad& \text{Vaccination rates}\\
&\mathbf{D} \in \mathbb{R}^{k\times k}, \mathbf{D}:=diag(D_1,...,D_k) &\quad& \text{Death rates due to other causes}\\
&\beta(t):\mathbb{R} \rightarrow \mathbb{R}^{k\times k}&\quad& \text{The (potentially time-dependent) contact rate}\\
&\gamma \in \mathbb{R}&\quad&\text{Recovery rate}\\
&\mathbf{M}\in\mathbb{R}^{k\times k}&\quad& \text{The age group transition matrix}&
\end{align*}

  where $\mathbf{M}$ has the structure
\begin{gather*}
\mathbf{M}=\begin{bmatrix}
-m_1^{-1} &      &        &        &    \\
 m_1^{-1} & -m_2^{-1} &        &        &    \\
     &  m_2^{-1} & -m_3^{-1}   &        &    \\
     &      & \ddots & \ddots &    \\
     &      &		 & m_{k-1}^{-1}& 0\\
\end{bmatrix}
\end{gather*}
\section{Data}
Our data source is \href{https://webarchive.nationalarchives.gov.uk/20140505133414/http://www.hpa.org.uk/Topics/InfectiousDiseases/InfectionsAZ/Rubella/EpidemiologicalData/}{Rubella Data}.

The immunisaton coverage data is from \href{https://webarchive.nationalarchives.gov.uk/20140505192935/http://www.hpa.org.uk/web/HPAweb&HPAwebStandard/HPAweb_C/1195733819251}{Immunisation Coverage}.

The confirmed rubella cases is in \href{https://webarchive.nationalarchives.gov.uk/20140505195855/http://www.hpa.org.uk/web/HPAweb&HPAwebStandard/HPAweb_C/1195733752351}{Rubella Cases by Age}.


From there,we could get access to the confirmed cases of rubella from 1996-2018 in different age groups, which will make our age structured SIRV model doable. Also, we could validate our model by those data.

We could use the proportion of female cases in a certain age group cases to predict how many females contract rubella in one year, and compare to that in SIR model without vaccine to see the importance of vaccine.

\end{document}
