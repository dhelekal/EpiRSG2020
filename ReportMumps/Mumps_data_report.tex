\documentclass[a4paper, 12pt]{article}

\usepackage[textheight=700pt,textwidth=480pt, bottom=80pt]{geometry}


% set font encoding for PDFLaTeX, XeLaTeX, or LuaTeX
\usepackage{ifxetex,ifluatex}
\if\ifxetex T\else\ifluatex T\else F\fi\fi T%
  \usepackage{fontspec}
\else
  \usepackage[T1]{fontenc}
  \usepackage[utf8]{inputenc}
  \usepackage{lmodern}
\fi

\newcommand{\MYhref}[3][blue]{\href{#2}{\color{#1}{#3}}}%

\usepackage{xcolor}
\usepackage[colorlinks = true,
            linkcolor = blue,
            urlcolor  = blue,
            citecolor = blue,
            anchorcolor = blue]{hyperref}
\usepackage{bbding} %checked
\usepackage{amsmath}

\begin{document}

\begin{center}
\large{\scshape University of Warwick}\\
{Msc in Mathematics of Systems}\\
	\vspace{0.75cm}
\textbf{Research Study Group}\\
	\Large{Mumps Data \& Model}\\
		\large{}
			\vspace{0.75cm}
			\normalsize{Francesca Basini, Melissa Iacovidou}
\end{center}


\section*{Data}
\subsection*{Most recent EU outbreaks}
\begin{itemize}
    \item \textbf{UK:} 2004-2019, mainly universities. (Gupta et al. 2005)
    \item \textbf{Ireland:} 2009, secondary schools and unis, probably due to no-vax movements and economic issues of the gov which failed vaccine coverage.
    \item \textbf{Ireland:} 2019-20, due to low uptake in the 90s, quite high in some counties.
    \item[-] Belgium: 2012, Ghent uni, only few cases.
    \item[-] The Netherlands: 2008-10, students, might be due to previous no-vax movements.
\end{itemize}


We decided to focus on UK and possibly Ireland.

\subsection*{UK data and general info}
\begin{itemize}
    \item \textbf{Wales and England:} confirmed cases.
    \href{https://www.gov.uk/government/publications/mumps-confirmed-cases}{Data Source.}\\
    \texttt{4 spreadsheet: total cases recorded quarterly and total annual cases stratified per county and age-class. Separated into 1995-2012 and 2013-2019. \CheckmarkBold} \\
    \texttt{data to be used only split into the age-groups mentioned in the Model section, we could disregard the values that are NK (Not Known) or take them into account as some sort of noise?}
    
    - More on Epidemiology, Surveillance and Control
    \href{https://www.gov.uk/government/collections/mumps-guidance-data-and-analysis}{link.}
    
    
    - General info on mumps condition from \href{https://www.nhs.uk/conditions/mumps/}{NHS website}
    
    
    - NHS Children vaccination coverage programme \href{https://digital.nhs.uk/data-and-information/publications/statistical/nhs-immunisation-statistics/england-2017-18}{data from 2017-18}
    
    
    - Mumps and risks in pregnancy \href{https://www.gov.uk/government/publications/mumps-risk-in-pregnancy-infection-in-healthcare-settings-and-mmr-vaccine/mumps-risk-in-pregnancy-infection-in-healthcare-settings-and-mmr-vaccine}{link to website}
    \item \textbf{Scotland: } we found some plots for quarterly records and for cases stratified by age structure at the following \href{https://www.hps.scot.nhs.uk/a-to-z-of-topics/mumps/}{link}
    
    \texttt{to access row data it is necessary to send a \textbf{FOI request.}} \textbf{$\Leftarrow$}
    
    - Child vaccination and immmunisation \href{https://www.isdscotland.org/Health-Topics/Child-Health/Immunisation/}{lint to website}
    
    \item \textbf{Ireland: } we have a set of pages, 
    \begin{itemize}
        \item[a)] \href{https://www.hpsc.ie/a-z/vaccinepreventable/mumps/}{reported cases}, 
        \item[b)] \href{https://www.hpsc.ie/a-z/vaccinepreventable/mmr-protectionagainstmeaslesmumpsrubella/}{MMR protection} 
        \item[c)] and some \href{https://www.hpsc.ie/a-z/vaccinepreventable/mumps/publications/annualreports/}{Annual reports}.
    \end{itemize} 
    \texttt{no info on how to get the raw data} \textbf{- ?}
\end{itemize}

\section*{Model [1]}

\subsubsection*{6 age groups:}
\begin{enumerate}
\item $0-4$ years (could possibly split to two groups and include the two
vaccinations)
\item $5-9$ years
\item $10-14$ years
\item $15-19$ years
\item $20-29$ years
\item $30+$ years
\end{enumerate}

\subsection*{ODE model}

\begin{align*}
\frac{dS_{1}}{dt}= & (1-\theta\sigma)\Lambda-\sum_{j=1}^{6}\beta_{1}c_{1j}S_{1}I_{j}-d_{1}S_{1}-\alpha_{1}S_{1}\\
\frac{dI_{1}}{dt}= & \sum_{j=1}^{6}\beta_{1}c_{1j}S_{1}I_{j}-(d_{1}+\gamma_{1}+\alpha_{1})I_{1}\\
\frac{dR_{1}}{dt}= & \theta\sigma\Lambda+\gamma_{1}I_{1}-d_{1}R_{1}-\alpha_{1}R_{1}
\end{align*}

\begin{align*}
\frac{dS_{k}}{dt}= & \alpha_{k-1}S_{k-1}-\sum_{j=1}^{6}\beta_{k}c_{kj}S_{k}I_{j}-d_{k}S_{k}-\alpha_{k}S_{k}\\
\frac{dI_{k}}{dt}= & \alpha_{k-1}I_{k-1}+\sum_{j=1}^{6}\beta_{1}c_{1j}S_{1}I_{j}-(d_{k}+\gamma_{k}+\alpha_{k})I_{k}\\
\frac{dR_{k}}{dt}= & \alpha_{k-1}R_{k-1}+\gamma_{k}I_{k}-d_{k}R_{k}-\alpha_{k}R_{k}
\end{align*}
where $k=2,3,4,5,6$ and $\alpha_{6}=0$.
$\Lambda=$ birth rate (influx of susceptibles)

$\theta=$ immunisation rate of vaccine

$\sigma=$ efficacy of vaccine

$\beta_{k}=$ probability of transmission per contact for age group
$k$

$c_{kj}=$ average number of contacts from age group $j$ to age group
$k$

$d_{k}=$ natural death rate

$\alpha_{k}=$ aging rate

$\gamma_{k}=$ recovery rate\\

\newpage
Two doses of the MMR vaccine ($88\%$ effective after two doses):
\begin{enumerate}
\item within a month of 1st birthday
\item before school (3 years \& 4 months)
\end{enumerate}
Can have a "catch-up" MMR vaccination up to the age of 18.
\begin{itemize}
\item people born from 1970-1979 may have only been vaccinated against measles
\item people born from 1980 to 1990 may not be protected against mumps
\end{itemize}

\section*{References}
\begin{enumerate}
\item Zhou, L., Wang, Y., Xiao, Y., \& Li, M. Y. (2019). Global dynamics of a discrete age-structured SIR epidemic model with applications to measles vaccination strategies. Mathematical biosciences, 308, 27-37.
\end{enumerate}


\end{document}
